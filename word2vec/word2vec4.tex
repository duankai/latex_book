\chapter{Word2Vec训练同义词模型}
\section{需求描述}
业务需求的目标是识别出目标词汇的同义词和相关词汇,如下为部分目标词汇(主要用于医疗问诊):
\begin{itemize}
    \item 尿
    \item 痘痘
    \item 发冷
    \item 呼吸困难 
    \item 恶心
\end{itemize}

数据源是若干im数据,那么这里我们选择google 的word2vec模型来训练同义词和相关词。

\section{数据处理}

数据处理考虑以下几个方面:
\begin{enumerate}
    \item 从hive中导出不同数据量的数据
    \item 过滤无用的训练样本(例如字数少于5)
    \item 准备自定义的词汇表
    \item 准备停用词表
\end{enumerate}

\section{工具选择}

选择python 的gensim库,由于先做预研,数据量不是很大,选择单机就好,
暂时不考虑spark训练。后续生产环境计划上spark。
\href{https://radimrehurek.com/gensim/models/word2vec.html}{详细的gensim中word2vec文档}

上述文档有关工具的用法已经很详细了,就不多说。

分词采用jieba。

\section{模型训练步骤简述}
\begin{enumerate}
    \item 先做分词、去停用词处理
    \begin{python}
        seg_word_line = jieba.cut(line, cut_all = True)
    \end{python}
    \item 将分词的结果作为模型的输入
    \begin{python}
        model = gensim.models.Word2Vec(
            LineSentence(source_separated_words_file), 
            size=200, window=5, min_count=5, alpha=0.02, 
            workers=4)
    \end{python}
    \item 保存模型,方便以后调用,获得目标词的同义词
    \begin{python}
        similary_words = model.most_similar(w, topn=10)
    \end{python}
\end{enumerate}

\section{重要调参目标}

比较重要的参数:
\begin{itemize}
    \item 训练数据的大小,当初只用了10万数据,训练出来的模型很不好,后边不断地将训练语料增加到800万,效果得到了明显的提升
    \item 向量的维度,这是词汇向量的维数,这个会影响到计算,理论上来说维数大一点会好。
    \item 学习速率
    \item 窗口大小
\end{itemize}

在调参上,并没有花太多精力,因为目测结果还好,到时上线使用前再仔细调整。

\section{模型的实际效果}
\begin{table}[h]
    \centering  % 显示位置为中间
    \caption{同义词}  % 表格标题
    \label{table1}  % 用于索引表格的标签
    %字母的个数对应列数,|代表分割线
    % l代表左对齐,c代表居中,r代表右对齐
    \begin{tabular}{|l|l|}  
        \hline
        目标词 & 同义词相关词 \\
        \hline  % 表格的横线
        尿     & 尿液,撒尿,尿急,尿尿有,尿到,内裤,尿意,小解,前列腺炎,小便 \\
        \hline
        痘痘   & 逗逗,豆豆,痘子,小痘,青春痘,红痘,长痘痘,粉刺,讽刺,白头\\
        \hline
        发冷   & 发烫,没力,忽冷忽热,时冷时热,小柴胡,头昏,嗜睡,38.9,头晕,发寒 \\
        \hline
        呼吸困难&气来,气紧,窒息,大气,透不过气,出不上,濒死,粗气,压气,心律不齐\\
        \hline
        恶心    &闷,力气,呕心,胀气,涨,不好受,不进,晕车,闷闷,精神\\
        \hline
    \end{tabular}
\end{table}

\section{可以跑的CODE}
\begin{python}
import codecs
import jieba
import gensim
from gensim.models.word2vec import LineSentence

def read_source_file(source_file_name):
    try:
        file_reader = codecs.open(source_file_name, 'r', 'utf-8',errors="ignore")
        lines = file_reader.readlines()
        print("Read complete!")
        file_reader.close()
        return lines
    except:
        print("There are some errors while reading.")

def write_file(target_file_name, content):
    
    file_write = codecs.open(target_file_name, 'w+', 'utf-8')
    file_write.writelines(content)
    print("Write sussfully!")
    file_write.close()

def separate_word(filename,user_dic_file, separated_file):
    print("separate_word")
    lines = read_source_file(filename)
    #jieba.load_userdict(user_dic_file)
    stopkey=[line.strip() for line in codecs.open('stopword_zh.txt','r','utf-8').readlines()]

    output = codecs.open(separated_file, 'w', 'utf-8')
    num = 0
    for line in lines:
        num = num + 1
        if num% 10000 == 0:
            print("Processing line number: " + str(num))
        seg_word_line = jieba.cut(line, cut_all = True)
        wordls = list(set(seg_word_line)-set(stopkey))
        if len(wordls)>0:
            word_line = ' '.join(wordls) + '\n'
        output.write(word_line)
    output.close()
    return separated_file
   

def build_model(source_separated_words_file,model_path):

    print("start building...",source_separated_words_file)
    model = gensim.models.Word2Vec(LineSentence(source_separated_words_file), size=200, window=5, min_count=5, alpha=0.02, workers=4)       
    model.save(model_path)
    print("build successful!", model_path)
    return model

def get_similar_words_str(w, model, topn = 10):
    result_words = get_similar_words_list(w, model)    
    return str(result_words)


def get_similar_words_list(w, model, topn = 10):
    result_words = []
    try:
        similary_words = model.most_similar(w, topn=10)
        print(similary_words)
        for (word, similarity) in similary_words:
            result_words.append(word)
        print(result_words)
    except:
        print("There are some errors!" + w)
        
    return result_words

def load_models(model_path):
    return gensim.models.Word2Vec.load(model_path)

if "__name__ == __main__()":
    filename = "d:\\data\\dk_mainsuit_800w.txt" #source file
    user_dic_file = "new_dict.txt" # user dic file
    separated_file = "d:\\data\\dk_spe_file_20170216.txt" # separeted words file
    model_path = "information_model0830" # model file
    
    #source_separated_words_file = separate_word(filename, user_dic_file, separated_file)
    source_separated_words_file = separated_file    # if separated word file exist, don't separate_word again
    build_model(source_separated_words_file, model_path)# if model file is exist, don't buile modl 

    model = load_models(model_path)
    words = get_similar_words_str('headache', model)
    print(words)
\end{python}