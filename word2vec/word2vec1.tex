\chapter{Word2Vec基本数学内容}
\section{Sigmod 函数}
Sigmod函数通常在二分类中应用。它将样本映射后投影在[0, 1]范围内,对应样本所属的类的概率。函数表达式如下所示:
\begin{equation}
    f(x) = \frac{1}{1+e^{-x}}
\end{equation}

具体的讨论可以参见:

\href{http://blog.csdn.net/chunyun0716/article/details/51580342}{Logistic Function AND Softmax Function}

\section{贝叶斯公式}
\begin{equation}
    P(A|B) = \frac{P(B|A)P(A)}{P(B)}
\end{equation}

可以参见贝叶斯分类等一系列文章:
\begin{enumerate}
    \item \href{http://blog.csdn.net/chunyun0716/article/details/51031055}{朴素贝叶斯分类}
    \item \href{http://blog.csdn.net/chunyun0716/article/details/51058948}{朴素贝叶斯算法的后验概率最大化含义}
    \item \href{http://blog.csdn.net/chunyun0716/article/details/51111864}{朴素贝叶斯法的参数估计}
\end{enumerate}

\section{Huffman 树和Huffman编码}
下边这篇博客写的很详细了,这里简单引用一些基本知识:

\href{http://blog.csdn.net/shuangde800/article/details/7341289}{哈夫曼树}


定义哈夫曼树之前先说明几个与哈夫曼树有关的概念:

\textbf{路径}: 树中一个结点到另一个结点之间的分支构成这两个结点之间的路径。

\textbf{路径长度}:路径上的分枝数目称作路径长度。

\textbf{树的路径长度}:从树根到每一个结点的路径长度之和。

\textbf{结点的带权路径长度}:在一棵树中,如果其结点上附带有一个权值,通常把该结点的路径长度与该结点上的权值                                                     之积称为该结点的带权路径长度(weighted path length)

\textbf{树的带权路径长度}:如果树中每个叶子上都带有一个权值,
则把树中所有叶子的带权路径长度之和称为树的带权路径长度。

一般来说,用$n(n>0)$个带权值的叶子来构造二叉树,
限定二叉树中除了这n个叶子外只能出现度为2的结点。
那么符合这样条件的二叉树往往可构造出许多颗,
其中带权路径长度最小的二叉树就称为\textbf{哈夫曼树}或\textbf{最优二叉树}.

\textbf{通过哈夫曼树来构造的编码称为哈弗曼编码(huffman code)}