\chapter{Spark下的word2vec模型训练}

\section{引言}
前边一节介绍了Word2Vec模型训练同义词,那么在大数据量的情况下,我们自然想到了用spark来进行训练下面就介绍我们是如何实现spark上的模型训练。

\section{分词}
模型训练的输入是分好词的语料,那么就得实现spark上的分词。

\begin{python}
def split(jieba_list, iterator):
    sentences = []
    for i in iterator:
        try:
            seg_list = []
            #out_str = ""
            s = ""
            for c in i:
                if not c is None:
                    s += c.encode('utf-8')
            id = s.split("__")[0]
            s = s.split("__")[1]
            wordList = jieba.cut(s, cut_all=False)
            for word in wordList:
	            out_str += word
	            out_str += " "
		        sentences.append(out_str)
        except:
            continue
    return sentences
\end{python}

\section{模型训练}

这里,直接用分词后的rdd对象作为输入

\begin{python}
   word2vec = Word2Vec().setNumPartitions(50)
   spark.sql("use jkgj_log")
   df = spark.sql("select label1_name,label2_name from mid_dim_tag ")
   df_list = df.collect()
   spark.sparkContext.broadcast(df_list)

   diagnosis_text_in = spark.sql("select main_suit,
                        msg_content from diagnosis_text_in where pt>='20170101'")
    
   inp = diagnosis_text_in.rdd.repartition(1200)
        .mapPartitions(lambda it: split(df_list,it))
        .map(lambda row: row.split(" "))

   model = word2vec.fit(inp)
\end{python}